\documentclass[a4paper, 11pt]{article}

\usepackage[utf8]{inputenc}
\usepackage[T1]{fontenc}
\usepackage[brazilian]{babel}

%%%%%%%%%%%%%%%%
%% Formatação %%

\usepackage{geometry}
\geometry{
    left = 1.5cm,
    right = 1.5cm,
    top = 1cm,
    bottom = 1.5cm,
    noheadfoot = true
}

\usepackage{enumitem}
\usepackage{float}
\usepackage{booktabs}
\usepackage{hyperref}

%%%%%%%%%%%%%%%
%% Documento %%

\title{Plano de Trabalho do Projeto Final de Graduação}
\author{Leonardo de Sousa Rodrigues e Tiago de Paula Alves}
\date{\today}

\begin{document}
    \maketitle

    \section{Objetivos}
    
        No estudo de rearranjo de genomas, considerando o contexto de representação de genomas por strings, um conjunto de problemas de interesse é o de Problemas de Partição de Strings. Tais problemas consistem em receber duas strings e encontrar formas de segmentar uma delas e rearranjar as partes criadas para formar a outra. Os problemas são diferenciados pelas restrições feitas à forma com que se pode segmentar ou rearranjar as partes. Além da reordenação, o problema pode ser considerado com as operações de inversão ou até a remoção de partes.

        Durante o decorrer do projeto, pretende-se estudar e desenvolver heurísticas para os Problemas de Partição de Strings e aplicá-las a uma base de dados a fim de comparar os desempenhos. Na sequência, serão desenvolvidas meta-heurísticas, que também terão desempenho mensurado e comparado.

    \section{Procedimento}

        O andamento do projeto será baseado no cronograma abaixo, com possíveis alterações do tempo dedicado a algumas atividades. A princípio, os dois estudantes participarão de todas as etapas do projeto e serão orientados em reuniões semanais.

    \section{Cronograma}

        \begin{table}[H]
            \centering
            \begin{tabular}{|c||c|c|c|c|c|c|c|c|c|c|}
                \hline
                 & \multicolumn{2}{c|}{Julho} & \multicolumn{2}{c|}{Agosto} & \multicolumn{2}{c|}{Setembro} & \multicolumn{2}{c|}{Outubro} & \multicolumn{2}{c|}{Novembro} \\
                \hline
                 & 01-15 & 16-31 & 01-15 & 16-31 & 01-15 & 16-30 & 01-15 & 16-31 & 01-15 & 16-30 \\
                \hline\hline
                1 & &*&*&*& & & & & & \\
                \hline
                2 & & &*&*&*& & & & & \\
                \hline
                3 & & &*&*&*& & & & & \\
                \hline
                4 & & & & &*&*&*& & & \\
                \hline
                5 & & & & & &*&*&*&*& \\
                \hline
                6 & & & &*&*& &*& &*& \\
                \hline
                7 & & & & &*& &*& &*& \\
                \hline
                8 & & & & &*& &*& &*& \\
                \hline
                9 & & & & & &*& &*& &*\\
                \hline
            \end{tabular}
            \caption{Cronograma previsto para o desenvolvimento do projeto.}
            \label{tab:cronograma}
        \end{table}

        \begin{enumerate}[itemsep=0pt]
            \item Revisão da literatura.
            \item Criação da base de dados.
            \item Desenvolvimento de heurísticas.
            \item Viabilidade das operações de reversão, inserção e deleção.
            \item Desenvolvimento de meta-heurísticas.
            \item Execução dos experimentos.
            \item Comparação dos resultados.
            \item Escrita do relatório.
            \item Revisão do relatório.
        \end{enumerate}

\end{document}
